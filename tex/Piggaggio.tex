\clearpage{\pagestyle{empty}\cleardoublepage}
\chapter{Sistemi di piggaggio delle condotte in pressione}\thispagestyle{empty} 
\chaptermark{Piggaggio}
Definizione di "line pigging" e "pig"
\section{Scopi del piggaggio}
	\begin{itemize}
		\item \textit{Pulizia della condotta}
		\item \textit{Batching}: pig per la separazione dei fluidi per vari scopi
		\item \textit{Ispezione interna}: smart pig o gauging (calibro)
		\item \textit{Spiazzamento della condotta}
	\end{itemize}
\section{Classificazione dei pig}
\begin{enumerate}
	\item \textbf{Assemblaggio}:
	\begin{itemize}
		\item A mandrino 
		\item A bullone singolo
		\item A telaio fisso: sia le tenute che gli elementi guida solo tenuti assieme da un singolo componente in poliiuretano
		\item A schiuma: i pig a schiuma di poliuretano con struttura aperto sono disponibili a diversa densità a seconda dell'applicazione richiesta; a volte possono essere rivestiti da poliuretano così da aumentare la resistenza a usura e la stabilità.
		\item Articolato
		\item Sfere
	\end{itemize}
	\item \textbf{Applicazioni}:
	\begin{itemize}
		\item "Bacthing pig"
		\item Di pulizia
		\item Magnetico
		\item Pig intelligente o "smart pig"
		\item A gel
	\end{itemize}
\end{enumerate}

\section{Scelta del pig idoneo in base alle operazioni richieste}

\subsection{Componenti del pig}
\subsubsection{Spazzole}
\subsubsection{Guarnizioni a tazza (unidirezionali)}
\subsubsection{Coppelle coniche (unidirezionali)}
\subsubsection{Dischi (bidirezionali)}
\subsubsection{Lame}
\subsubsection{Magneti}

\subsection{Obiettivo del piggaggio}

\subsection{Selezione del pig}
\subsubsection{Pulizia della condotta}
\subsubsection{Spostamento fluidi}
\subsubsection{Spiazzamento di liquidi da linee gas umido}
\subsubsection{Rimozione di detriti ferrosi}
\subsubsection{Applicazione degli inibitori di corrosione}
\subsubsection{Pulizia pre-ispezione}
\subsubsection{Pulizia delle condotte con corrosione interna nota}
\subsubsection{Pulizia delle condotte internamente rivestite}
\subsubsection{Piggaggio delle condotte offshore}


\section{Design del sistema di condotte per il piggaggio}
Il sistema deve essere progettato in modo tale da evitare imprevisti nelle operazioni di piggaggio. Di conseguenza ci sono degli importanti elementi di design da tenere in considerazione.
\subsection{Lunghezza del cammino del pig}
\subsection{Raggio di curvatura}
\subsection{Tipologia di valvole presenti}
\subsection{Indicatori di passaggio del pig}
\subsection{Raccordi}
\subsection{Condotte laterali}
\subsection{Raggio interno di condotta}
\subsection{Condotte a doppio diametro}


\section{Procedure di lancio e recupero del pig}

\subsection{Condotte a liquido}
\subsubsection{Procedure di lancio pig per condotte a liquido}
\subsubsection{Procedure di recupero pig per condotte a liquido}

\subsection{Condotte a gas}
\subsubsection{Procedure di lancio pig per condotte a gas}
\subsubsection{Procedure di recupero pig per condotte a gas}

\subsection{Esempi di configurazione di sistemi di piggaggio}
\subsubsection{Lancio verticale}
\subsubsection{Lancio standard del pig}
\subsubsection{Lancio doppio o duale}
\subsubsection{Sistema di batch automatico}
\subsubsection{Lancio inclinato}
\subsubsection{Lancio combinato pig/sfera}
\subsubsection{Sistemi di piggaggio automatico}
\subsubsection{Lancio del pig di ispezione in linea}

\section{Piggaggio di una condotta onstream con mezzo gassoso o idrocarburi raffinati}
\subsection{Scelta del pig}
\subsection{Velocià}
\subsection{Frequenza}
\subsection{Prestazioni}

\section{Bloccaggio del pig o "stuck pig"}
\subsection{Soluzioni}
\subsection{Fasi operative}
\subsection{Sviluppo futuro}