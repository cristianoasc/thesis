\clearpage{\pagestyle{empty}\cleardoublepage}
\chapter*{Introduzione}\thispagestyle{empty} 
\addcontentsline{toc}{chapter}{Introduzione}
\rhead[\fancyplain{}{\bfseries\scshape
Introduzione}]{\fancyplain{}{\thepage}}
\lhead[\fancyplain{}{\thepage}]{\fancyplain{}{\bfseries\scshape
Introduzione}}
%\thispagestyle{empty}
%Inquadramento argomento tesi\\
%Descrizione struttura tesi\\
%Conclusioni preliminari\\
Il settore dell'\textit{Oil\&{Gas}} provvede oggi a gran parte del fabbisogno energetico mondiale. Gli idrocarburi assumono quindi enorme importanza nell'economia e nella geopolitica moderna grazie al ruolo fondamentale di fonte energetica. Nell'ambito dell'Industria Petrolifera si parla spesso di \textit{Flow Assurance}, concetto coniato dalla Petrobras all'inizio degli anni '90, che lega in modo biunivoco la resa economica della produzione di idrocarburi con il mantenimento di condizioni ottimali di trasporto in condotta dal giacimento al punto di vendita.\\
I surfactanti risultano essere una delle soluzioni più idonee e versatili per migliorare la produzione di idrocarburi. Gli schiumogeni sono impiegati in tutte le fasi, dalle operazioni di perforazione, al recupero assistito di petrolio o gas , fino al trasporto e il trattamento. Negli ultimi anni gli schiumogeni sono stati al centro di applicazioni per l'attenuamento del battente idrostatico di pozzi a gas caratterizzati da ingenti produzioni di acqua. L'incapacità di un pozzo di non spiazzare i liquidi al suo interno porta alla generazione di una forte contropressione che tende a opporre resistenza al normale deflusso del gas dal giacimento al \textit{tubing}. I surfactanti permettono la riduzione della tensione superficiale dei liquidi accumulati e il loro spiazzamento senza l'immissione di energia dall'esterno.\\
Nell'ambito della produzione di gas naturale il problema delle contropressioni generate da liquidi non riguarda solamente i pozzi; anche le condotte possono essere interessate da accumuli localizzati di acqua di condensa o di produzione. L'\textit{hold-up} dei liquidi provoca restringimenti lungo la condotta che si traducono in perdite di carico, diminuendo così la portata di gas in linea.\\
Per ovviare al problema, sono operati periodicamente in condotta dei lanci di pig, dispositivi in grado di svolgere numerose operazioni, tra cui quella di pulizia e di spiazzamento dei fluidi stagnanti. Il piggaggio della linea presenta problemi legati al rischio di blocco dello strumento in linea e alla predisposizione della rete al passaggio del pig.\\
A fronte di tali considerazioni, l'impiego di schiumogeni per lo spiazzamento delle condotte orizzontali può ovviare agli svantaggi relativi al piggaggio della. Lo schiumogeno immesso in condotta raggiunge le zone di accumulo, agisce sul fluido e consente il trascinamento dello stesso tramite la corrente gassosa.\\
L'obiettivo di questo lavoro è quello di valutare l'efficacia dei tensioattivi in orizzontale, in presenza di fenomeni di \textit{hold-up} in linea. Il test è stato effettuato durante l'esperienza di tirocinio formativo e di orientamento presso la Edison S.p.A. nella sede operativa di Sambuceto (CH). Per l'applicazione ha interessato la linea di Verdicchio, appartenente al polo produttivo di San Giorgio Mare (FM). Le operazioni sono state svolte in collaborazione con la Chimec S.p.A., che ha fornito supporto tecnico e i prodotti chimici utilizzati. L'efficacia è stata confermata tramite il controllo dei parametri di produzione sul campo fino allo spiazzamento della condotta e il monitoraggio degli stessi nelle settimane successive.\\
Il lavoro di tesi si struttura in cinque capitoli principali. Il primo capitolo tratta le condizioni di regime di flusso multifase in condotta e le perdite di carico correlate. Nel secondo capitolo si discutono le proprietà fisico-chimiche dei tensioattivi, delle schiume e dell'applicazione di schiumogeni a fondo pozzo. Nel terzo e nel quarto capitolo si affrontano gli impianti di superficie per il trattamento del gas naturale e le relative tecniche per la pulizia e il mantenimento delle condotte. Nell'ultimo capitolo sono invece esposte le procedure operative dell'applicazione di schiumogeno in linea orizzontale effettuata presso il polo di San Giorgio Mare e i risultati avuti nel breve-medio termine.\\
L'impostazione del seguente lavoro è stata scelta accuratamente al fine di esporre in modo semplice e chiaro le conoscenze tecniche di base utili alla totale comprensione dell'applicazione in esame. L'importanza delle competenze tecniche e teoriche possono favorire il processo di ottimizzazione del metodo, potenzialmente un importante soluzione alternativa al piggaggio tradizionale delle condotte a gas.
\clearpage{\pagestyle{empty}\cleardoublepage}

