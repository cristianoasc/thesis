\chapter*{Sommario}\thispagestyle{empty}    %crea l'introduzione (un capitolo
                                        %   non numerato)
%%%%%%%%%%%%%%%%%%%%%%%%%%%%%%%%%%%%%%%%%imposta l'intestazione di pagina
%\rhead[\fancyplain{}{\bfseries
%Sommario}]{\fancyplain{}{\bfseries\thepage}}
%\lhead[\fancyplain{}{\bfseries\thepage}]{\fancyplain{}{\bfseries
%Sommario}}
%%%%%%%%%%%%%%%%%%%%%%%%%%%%%%%%%%%%%%%%%aggiunge la voce Introduzione
                                        %   nell'indice
L'industria di processo legata al settore dell'\textit{Oil\&Gas} risente fortemente delle condizioni di trasporto in condotta delle risorse prodotte. Il piggaggio rappresenta oggi l'operazione consueta per la pulizia e lo spiazzamento delle linee. I pig presentano numerosi problemi, come il rischio di blocco  o l'impossibilità di utilizzo a causa del design di rete. Gli schiumogeni, impiegati in numerosi ambiti nella produzione di idrocarburi, posso operare nello spiazzamento delle condotte a gas orizzontali. Il metodo è del tutto simile all'iniezione di \textit{foamer} nel \textit{Gas Well Deliquification}, dove si sfrutta l'abbassamento della tensione superficiale per favorire il trascinamento della fase liquida a fondo pozzo da parte della corrente gassosa. Il test di efficacia è stato effettuato da Edison S.p.A., in collaborazione con Chimec S.p.A., nella linea a gas di Verdicchio, appartenente al polo produttivo di San Giorgio Mare (FM). I surfactanti hanno agito in sole due ore dall'applicazione, con lo spiazzamento di 12,74 m\ap{3} di acqua, un aumento della produzione di gas del 5\% e una diminuzione della contropressione di 9,5 bar. I tensioattivi dimostrano grandi potenzialità anche in questo ambito e la sfida futura è rappresentata dalla ricerca e lo sviluppo di procedure operative consolidate.
%%%%%%%%%%%%%%%%%%%%%%%%%%%%%%%%%%%%%%%%%non numera l'ultima pagina sinistra
\clearpage{\pagestyle{empty}\cleardoublepage}