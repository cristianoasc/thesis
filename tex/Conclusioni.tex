\chapter*{Conclusioni}\thispagestyle{empty} 
\addcontentsline{toc}{chapter}{Conclusioni}
\rhead[\fancyplain{}{\bfseries\scshape
Conclusioni}]{\fancyplain{}{\thepage}}
\lhead[\fancyplain{}{\thepage}]{\fancyplain{}{\bfseries\scshape
Conclusioni}}
Oltre alle valutazioni sugli effetti a breve termine, è utile soffermarsi sugli effetti positivi a lungo termine dell'applicazione del foamer. La possibilità di operare con condotte caratterizzate da contropressioni ridotte garantisce vantaggi su lunga durata, quindi influisce sui costi di manutenzione straordinaria. L'utilizzo di schiuma permette inoltre una veloce e rapida pulizia della condotta da residui solidi di produzione e diminuisce l'eventuale corrosione a opera di batteri solfato-riduttori provenienti da giacimenti acidi di gas.\\
Il pig chimico rappresenta inoltre un'interessante alternativa al tradizionale pig meccanico. Il batch di schiuma può essere utilizzato dove la condotta non è piggabile (per motivi geometrici o di impianto) e risulta essere un intervento meno invasivo rispetto a quello meccanico, prevenendo eventuali danni causati dal trascinamento del pig durante lo spiazzamento dell'acqua. I guadagni maggiori si hanno in termini di tempo e denaro, operando nell'arco di alcune ore e con l'ausilio di meno personale sul posto.\\
L'applicazione del pig chimico presenta oggi due incognite: la definizione di procedure operative consolidate e l'efficacia su condotte di diametri maggiori a 6".
\clearpage{\pagestyle{empty}\cleardoublepage}
