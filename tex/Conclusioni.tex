\chapter*{Conclusioni}\thispagestyle{empty} 
\addcontentsline{toc}{chapter}{Conclusioni}
\rhead[\fancyplain{}{\bfseries\scshape
Conclusioni}]{\fancyplain{}{\thepage}}
\lhead[\fancyplain{}{\thepage}]{\fancyplain{}{\bfseries\scshape
Conclusioni}}
Obiettivo principale del lavoro svolto è la conferma dell'efficacia di schiumogeni per lo spiazzamento di liquidi in condotta, sulla falsa riga delle applicazioni di tensioattivi per i pozzi affetti da \textit{liquid loading}. Nello specifico si è cercato di valutare l'efficacia contro i fenomeni di \textit{hold-up} in condizioni orizzontali, sia nell'immediato che nelle settimane successive all'applicazione. Il test è stato confrontato con operazioni di piggaggio precedenti per capire quanto questa tecnologia possa essere percorribile in un futuro prossimo.\\
Il test ha portato allo spiazzamento della condotta in poche ore, con un volume di acqua confluito a valle che rispecchia le quantità di liquido rimosso dai precedenti piggaggi della linea. La contropressione in linea ha subito una brusca diminuzione e ha raggiunto valori minimi dopo solo alcune ore dall'esecuzione del test.\\
Nelle settimane successive all'operazione, il monitoraggio dei parametri di produzione ha evidenziato che i valori di contropressione sono rimasti costanti anche a distanza di settimane, testimoniando il fatto di come il metodo abbia migliorato le condizioni di trasporto nel breve-medio periodo.
La produzione di gas è aumentata grazie alle minori cadute di pressione nella linea a valle. Oltre all'incremento della portata di gas, si è ottenuto una maggiore costanza nei valori di produzione. Ciò costituisce un buon segnale sui benefici che una condotta caratterizzata da basse cadute di pressione può portare ai relativi pozzi a gas. L'ottimizzazione della produzione ha visto una nuova configurazione delle valvole regolatrici, consentendo di far lavorare i pozzi a minori valori di contropressione. Anche le altre soluzioni per il miglioramento produttivo come i \textit{foamer stick} non si sono rese più necessarie dopo lo spiazzamento della condotta.
In generale l'acqua di produzione è aumentata ma il breve tempo necessario all'operazione ha evitato problemi legati all'interruzione prolungata della linea. Il piggaggio necessita di maggiori tempi operativi e lo \textit{shut-down}  richiesto porta all'innalzamento delle pressioni a testa pozzo. All'avvio della produzione, le maggiori portate di gas sono accompagnate da importanti cuscini d'acqua, ingestibili dai separatori di campo e gli avvallamenti sono nuovamente invasi dal liquido.\\
I risultati si raccordano con i presupposti teorici iniziali. L'\textit{hold-up} in condotta non è stato soltanto attenuato, bensì totalmente abbattuto e inibito nelle successive settimane dal test.  L'ultima considerazione formulata trova conferma nei valori della contropressione in linea, la quale in seguito non si è mai innalzata e ha mantenuto costante il suo valore nella la fase di monitoraggio post-test. Lo spiazzamento totale d'acqua non è avvenuto in maniera costante come ipotizzato ma si è realizzato in soluzione unica, offrendo dei nuovi scenari per applicazioni di questo genere. Le condizioni di produzione sono migliorate come previsto, dato il forte legame tra cadute di pressione e portata in condotta. La regolarità negli andamenti di produzione costituisce un buon risultato in termini di rendimenti del pozzo. L'irregolarità della portata di gas è associata nel termine vita di un pozzo a fenomeni di \textit{liquid loading}: una minore contropressione a monte, in questo caso garantita dal nuovo settaggio della PCV, garantisce energia sufficiente al pozzo per evitare ulteriori incrementi della colonna idrostatica a fondo pozzo. Le precedenti condizioni favoriscono il trasporto d'acqua in superficie con cuscini d'acqua più piccoli, quindi più facilmente gestibili dai dispositivi di separazione presenti sull'area pozzo.\\
I tempi attesi per la riuscita dell'applicazione sono andati al di là di qualunque buona aspettativa. Lo schiumogeno ha agito come batch unico e compatto in linea, ridefinendo l'applicazione in questo lavoro come "pig chimico". La soluzione presenta notevoli vantaggi rispetto alla all'immissione di schiumogeno. L'attivazione del meccanismo di spiazzamento non avviene al raggiungimento della concentrazione voluta in acqua, ma la schiuma è generata in corrispondenza del punto di lancio. Il pig chimico non richiede l'installazione di pompe di iniezione con trattamento in continuo, associate a costi operativi maggiori. La soluzione unica si concretizza in una riduzione di costi ed è anche molto più accattivante per la semplicità di esecuzione.\\
Oltre alle valutazioni sull'efficacia del metodo su breve-medio periodo, è utile soffermarsi sui potenziali benefici a lungo termine che interessano l'impianto. La possibilità di operare con condotte caratterizzate da contropressioni ridotte garantisce vantaggi su lunga durata, quindi influisce sui costi di manutenzione straordinaria. L'utilizzo di schiuma permette inoltre una veloce e rapida pulizia della condotta da residui solidi di produzione e diminuisce l'eventuale corrosione a opera di batteri solfato-riduttori provenienti da giacimenti acidi di gas. Data la conformazione del pig chimico, l'efficacia di pulizia è garantita su tutta la parete interna della condotta.\\
Questa applicazione rappresenta di per certo un'interessante alternativa al tradizionale pig meccanico. Il batch di schiuma può essere utilizzato dove la condotta non è piggabile (per motivi geometrici o di impianto) e risulta essere un intervento meno invasivo rispetto a quello meccanico, prevenendo eventuali danni causati dal trascinamento del pig durante lo spiazzamento dell'acqua. I guadagni maggiori si hanno in termini di tempo e denaro, operando nell'arco di alcune ore e con l'ausilio di meno personale sul posto.\\
La soluzione può essere vista anche come alternativa al piggaggio relativo al trattamento chimico in condotta. In caso di trattamento spinto da realizzarsi su tutta la parete interna, i metodi oggi adottati sono l'immissione di batch chimici accompagnati da pig di separazione a tenuta idraulica oppure l'utilizzo di scovoli dotati di serbatoio interno al mandrino e ugelli per l'irrorazione del prodotto in linea. Il pig chimico può essere associato ad altri prodotti, consentendo anche l'ottima efficacia di trattamento chimico in termini di tempo e senza l'impiego di dispositivi meccanici.\\
A causa dello svolgimento del test su un'unica condotta non sono disponibili ulteriori indicazioni. La carenza di esperienza rappresenta oggi un forte limite per lo sviluppo della tecnica. La reologia di una schiuma è influenza da notevoli parametri e la raccolta di maggiori informazioni potrebbe fornire ulteriori dettagli per l'interpretazione del problema, di per se complesso.\\
La prima incognita è associata all'efficacia del pig chimico in condotte di diversa dimensione da quella trattata in questo lavoro. In particolare le condotte caratterizzate da diametri importanti potrebbero trovare maggiori difficoltà nell'impiego del batch, dove i volumi di acqua in condotta sono ingenti e di conseguenza le quantità di schiuma da immettere sono enormi.\\
Per la condotta su cui è stato condotto il test non è stata definita una procedura operativa consolidata. Il \textit{know-how} relativo alla tecnica può essere esteso solo grazie a ulteriori prove sul campo. Tali test possono essere effettuati solo al ristabilirsi delle condizioni di contropressione precedenti, le quali continuano a rimanere costanti a distanza di mesi.\\
Nonostante l'evidente efficacia e i numerosi vantaggi rispetto al piggaggio tradizionale, la vera sfida per il futuro è rappresentata dalla generalizzazione del metodo nei diversi impianti in cui si richiede l'applicazione. Il design del pig chimico deve passare dalla progettazione di prodotti chimici specifici, la creazione di dispositivi di generazione e lancio del batch e il mantenimento della reologia della schiuma al variare delle condizioni di flusso, pressione e temperatura in condotta.